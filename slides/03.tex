% Written in 2015 by Manuel Pégourié-Gonnard
% SPDX-License-Identifier: CC-BY-SA-4.0

\documentclass{mpg-ep-slides}

\author[MPG]{Manuel Pégourié-Gonnard\\
  \href{mailto:mpg@elzevir.fr}{{mpg@elzevir.fr}}
}
\institute[ARM]{\normalsize ARM France - IoT - mbed TLS}
\license{%
  \url{https://github.com/mpg/cours-tls} \\[.5ex]
  \href{https://creativecommons.org/licenses/by-sa/4.0/}{CC-BY-SA 4.0}
}

\title{Cours de cryptologie appliquée de l'EPITA \\ TLS - partie 3}
\date{26 novembre 2015}

\begin{document}

\lictitle

\section{Compléments}

\begin{frame}{Actu : Lucky 13 frappe Amazon s2n}
  \begin{center}
    \url{https://eprint.iacr.org/2015/1129}
  \end{center}
\end{frame}

\begin{frame}{Références complémentaires - TLS}
  \begin{itemize}
    \item Rappel : RFC 5246 TLS 1.2, RFC 7525 recommandations pratiques, RFC
      7457 attaques
    \item \url{https://www.trustworthyinternet.org/ssl-pulse/} statistiques
      sur les serveur HTTPS populaires
    \item \url{https://www.ssllabs.com/ssltest/} testez votre serveur !
  \end{itemize}
\end{frame}

\begin{frame}{Références crypto elliptique - en ligne}
  \begin{itemize}
    \item Courte introduction aux courbes et à des techniques d'implémentation
      efficace : \url{https://www.imperialviolet.org/2010/12/04/ecc.html}
    \item Intro plus détaillé avec code Python :
      \url{http://andrea.corbellini.name/2015/05/17/elliptic-curve-cryptography-a-gentle-introduction/}
    \item Un article retraçant l'histoire et pourquoi (on croit que) ça
      marche, avec un des inventeurs :
      \url{http://www.sciencedirect.com/science/article/pii/S0022314X09000481}
    \item Une version gratuite des standards :
      \url{http://www.secg.org/sec1-v2.pdf}
  \end{itemize}
\end{frame}

\begin{frame}{Références crypto elliptique - livres}
  \begin{itemize}
    \item Koblitz, \emph{A Course in Number Theory and Cryptogrphy}, Springer,
      GTM 114. Tout le bagage mathématique nécessaire, et plus encore, en 200
      pages (dont 40 sur les courbes elliptiques), avec une approche assez
      pratique (complexité algorithmique) pour un livre de maths.
    \item Hankerson, Menezes, Vanstone, \emph{Guide to Elliptic Curve
        Cryptography}, Springer. La bible de l'implémenteur il y a 10 ans,
      quelques manques depuis (side-channels, nouvelles formes de courbes).
    \item Cohen, Frey (Eds), \emph{Handbook of Elliptic and Hyperelliptic
        Curve Cryptography}, Chapman \& Hall. Juste \emph{la} référence sur
      tous les aspect théoriques.
  \end{itemize}
\end{frame}

\section{Pratique}

\begin{frame}{Projets}
  \begin{center}
    \url{https://github.com/mpg/cours-tls}
  \end{center}
\end{frame}

\end{document}
