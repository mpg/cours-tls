\documentclass{mpg-ep-slides}

\author[MPG]{Manuel Pégourié-Gonnard}
\institute[ARM]{\normalsize ARM France - IoT - mbed TLS}
\title{Cours de cryptologie appliquée de l'EPITA \\ TLS - partie 1}
\date{19 novembre 2015}

\begin{document}

\maketitle

\toc

\section{Introduction}
\tocsect

\begin{frame}{TLS ça fait quoi ?}
  Ça assure la sécurité des \emph{communications} :
  \begin{description}
    \item[Confidentialité] Un attaquant qui peut tout écouter ne peut rien
      apprendre sur les données échangées, à part peut-être leur longueur.
    \item[Intégrité] Un attaquant qui modifie les données en transit sera
      détecté.
    \item[Authentification] optionnelle d'une ou des deux parties :
      l'attaquant ne peut pas se faire passer pour quelqu'un d'autre.
  \end{description}
\end{frame}

\begin{frame}{Quelles menaces ?}
  RFC 3553 section 3 : the Internet threat model (2003)

  \begin{columns}
    \column[t]{.5\textwidth}
    \begin{block}{Attaquant réseau passif}
      \begin{itemize}
        \item Ne fait qu'écouter
        \item Attaque la confidentialité
        \item<2-> Échelle (perpass)
        \item<3-> Attaques « du futur »
      \end{itemize}
    \end{block}

    \begin{block}{Tout le reste}<4->
      Bugs, sécurité du poste local, utilisateurs, etc.
    \end{block}

    \column[t]{.5\textwidth}
    \begin{block}{Attaquant réseau actif}
      \begin{itemize}
        \item Peut modifier les messages
        \item Attaque l'authenticité, l'intégrité, la confidentialité
        \item Man-in-the-middle (MitM)
        \item<2-> Plus coûteux (ciblé ?)
      \end{itemize}
    \end{block}

    \begin{block}{Canaux auxiliaires}<4->
    \end{block}
  \end{columns}
\end{frame}

\begin{frame}{Ça s'utilise où ?}
  \begin{block}{Couche}
    \begin{description}
      \item[Application] HTTP, IMAP, SMTP, XMPP, \visible<2->{RTSP}
      \item[Session] TLS, \visible<2->{DTLS}
      \item[Transport] TCP, \visible<2->{UDP}
      \item[Internet] IPv4, IPv6
      \item[Lien] tout ce que vous voulez
    \end{description}
  \end{block}

  \begin{block}{Modes}
    \begin{itemize}
      \item avec port dédié : HTTP/80 \textrightarrow{} HTTPS/443
      \item sur le même port : STARTTLS (IMAP, SMTP, XMPP)
    \end{itemize}
  \end{block}
\end{frame}

\begin{frame}{Versions}
  \begin{center}
    \begin{tabular}{rcccl}
      \toprule
      Nom & année & RFC & die-die-die & « vraie » version \\
      \midrule
      SSL 1.0 & -- & -- & mort-né & 1.0 \\
      SSL 2.0 & 1995 & -- & 6176 (2011) & 2.0 \\
      SSL 3.0 & 1996 & 6101 & 7568 (2015) & 3.0 \\
      TLS 1.0 & 1999 & 2246 & -- & 3.1 \\
      TLS 1.1 & 2006 & 4346 & -- & 3.2 \\
      DTLS 1.0 & 2006 & 4347 & -- & 3.2 \\
      TLS 1.2 & 2008 & 5246 & -- & 3.3 \\
      DTLS 1.2 & 2012 & 6347 & -- & 3.3 \\
      \bottomrule
    \end{tabular}
  \end{center}
\end{frame}

\begin{frame}{La réalité}
  En vrai, il faut lire un peu plus de RFC\dots

  \begin{description}
    \item[Sécurité] 5746 secure renegotiation, 7627 session hash, 7366
      encrypt-then-mac, 7507 downgrade SCSV,
    \item[Algos] (300+ suites définies !) : 4279 PSK, 4492 ECC, 5054 SRP, 5288
      AES-GCM, 5289 ECC-AES-GCM, 5487 PSK-AES-GCM, 5489 ECDHE-PSK, 6655
      AES-CCM, 7251 ECC-AES-CCM, \dots
    \item[Fonctionnalités] 6066 extensions diverses, 5878 autorisation, 6520
      heartbeat, 5077 tickets de session, 7250 raw public key, 6091 OpenPGP,
      7301 ALPN, \dots
    \item[Usage] 7457 attaques, 7525 bonnes pratiques générales, 7590 XMPP,
      d'autres à venir
  \end{description}

  Cf les registres sur iana.org.
\end{frame}

\begin{frame}{Plus de réalité}
  En 2015, sur les sites les plus populaires accessibles en HTTPS
  \begin{itemize}
    \item environ un tiers considérés sûrs ;
    \item 14~\% acceptent des algos peu sûrs ;
    \item 31~\% acceptent SSL 3.0 ;
    \item 69~\% offrent TLS 1.2 ;
    \item 75~\% offrent la \emph{forward secrecy} ;
    \item 99,9~\% ont un certificat avec un clé assez grande.
  \end{itemize}

  Source : \url{https://www.trustworthyinternet.org/ssl-pulse/}

  \medskip

  Sur le million de sites les plus populaires, en 2014, seuls 45~\%
  accessible en HTTPS\dots
\end{frame}
\end{document}
