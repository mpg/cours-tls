% Written in 2015 by Manuel Pégourié-Gonnard
% SPDX-License-Identifier: CC-BY-SA-4.0

\documentclass{mpg-ep-slides}

\author[MPG]{Manuel Pégourié-Gonnard\\
  \href{mailto:mpg@elzevir.fr}{{mpg@elzevir.fr}}
}
\institute[ARM]{\normalsize ARM France - IoT - mbed TLS}
\license{%
  \url{https://github.com/mpg/cours-tls} \\[.5ex]
  \href{https://creativecommons.org/licenses/by-sa/4.0/}{CC-BY-SA 4.0}
}

\title{Cours de cryptologie appliquée de l'EPITA \\ TLS - partie 1}
\date{19 novembre 2015}

\begin{document}

\lictitle

\section{Introduction}
\tocsect

\begin{frame}{TLS ça fait quoi ?}
  Ça assure la sécurité des \emph{communications} :
  \begin{description}
    \item[Confidentialité] Un attaquant qui peut tout écouter ne peut rien
      apprendre sur les données échangées, à part peut-être leur longueur.
    \item[Intégrité] Un attaquant qui modifie les données en transit sera
      détecté.
    \item[Authentification] optionnelle d'une ou des deux parties :
      l'attaquant ne peut pas se faire passer pour quelqu'un d'autre.
  \end{description}
\end{frame}

\begin{frame}{Quelles menaces ?}
  RFC 3553 section 3 : the Internet threat model (2003)

  \begin{columns}
    \column[t]{.5\textwidth}
    \begin{block}{Attaquant réseau passif}
      \begin{itemize}
        \item Ne fait qu'écouter
        \item Attaque la confidentialité
        \item<2-> Échelle (perpass)
        \item<3-> Attaques « du futur »
      \end{itemize}
    \end{block}

    \begin{block}{Tout le reste}<4->
      Man-in-the-Browser, bugs, sécurité du poste local, utilisateurs, etc.
    \end{block}

    \column[t]{.5\textwidth}
    \begin{block}{Attaquant réseau actif}
      \begin{itemize}
        \item Peut modifier les messages
        \item Attaque l'authenticité, l'intégrité, la confidentialité
        \item Man-in-the-middle (MitM)
        \item<2-> Plus coûteux (ciblé ?)
      \end{itemize}
    \end{block}

    \begin{block}{Canaux auxiliaires}<4->
    \end{block}
  \end{columns}
\end{frame}

\begin{frame}{Ça s'utilise où ?}
  \begin{block}{Couche}
    \begin{description}
      \item[Application] HTTP, IMAP, SMTP, XMPP, \visible<2->{RTSP}
      \item[Session] TLS, \visible<2->{DTLS}
      \item[Transport] TCP, \visible<2->{UDP}
      \item[Internet] IPv4, IPv6
      \item[Lien] tout ce que vous voulez
    \end{description}
  \end{block}

  \begin{block}{Modes}
    \begin{itemize}
      \item avec port dédié : HTTP/80 \textrightarrow{} HTTPS/443
      \item sur le même port : STARTTLS (IMAP, SMTP, XMPP)
    \end{itemize}
  \end{block}
\end{frame}

\begin{frame}{Versions}
  \begin{center}
    \begin{tabular}{rcccl}
      \toprule
      Nom & année & RFC & die-die-die & « vraie » version \\
      \midrule
      SSL 1.0 & -- & -- & mort-né & 1.0 \\
      SSL 2.0 & 1995 & -- & 6176 (2011) & 2.0 \\
      SSL 3.0 & 1996 & 6101 & 7568 (2015) & 3.0 \\
      TLS 1.0 & 1999 & 2246 & -- & 3.1 \\
      TLS 1.1 & 2006 & 4346 & -- & 3.2 \\
      DTLS 1.0 & 2006 & 4347 & -- & 3.2 \\
      TLS 1.2 & 2008 & 5246 & -- & 3.3 \\
      DTLS 1.2 & 2012 & 6347 & -- & 3.3 \\
      \bottomrule
    \end{tabular}
  \end{center}
\end{frame}

\begin{frame}{La réalité}
  En vrai, il faut lire un peu plus de RFC\dots

  \begin{description}
    \item[Sécurité] 5746 secure renegotiation, 7627 session hash, 7366
      encrypt-then-mac, 7507 downgrade SCSV,
    \item[Algos] (300+ suites définies !) : 4279 PSK, 4492 ECC, 5054 SRP, 5288
      AES-GCM, 5289 ECC-AES-GCM, 5487 PSK-AES-GCM, 5489 ECDHE-PSK, 6655
      AES-CCM, 7251 ECC-AES-CCM, \dots
    \item[Fonctionnalités] 6066 extensions diverses, 5878 autorisation, 6520
      heartbeat, 5077 tickets de session, 7250 raw public key, 6091 OpenPGP,
      7301 ALPN, \dots
    \item[Usage] 7457 attaques, 7525 bonnes pratiques générales, 7590 XMPP,
      d'autres à venir
  \end{description}

  Cf les registres sur iana.org.
\end{frame}

\begin{frame}{Plus de réalité}
  En 2015, sur les sites les plus populaires accessibles en HTTPS
  \begin{itemize}
    \item environ un tiers considérés sûrs ;
    \item 14~\% acceptent des algos peu sûrs ;
    \item 31~\% acceptent SSL 3.0 ;
    \item 69~\% offrent TLS 1.2 ;
    \item 75~\% offrent la \emph{forward secrecy} ;
    \item 99,9~\% ont un certificat avec un clé assez grande.
  \end{itemize}

  Source : \url{https://www.trustworthyinternet.org/ssl-pulse/}

  \medskip

  Sur le million de sites les plus populaires, en 2014, seuls 45~\%
  accessible en HTTPS\dots
\end{frame}


\section[Couche record]{Généralités sur la couche record}
\tocsect

\begin{frame}{Les couches de TLS}
  \begin{center}
    \renewcommand\arraystretch{1.5}
    \begin{tabular}{|c|c|c|c|}
      \hline
      Handshake & ChangeCipherSpec & Alert &
      \multicolumn{1}{|c|}{ \color{gray} HTTP, SMTP, etc. } \\ \hline
      \multicolumn{4}{|c|}{Record} \\ \hline
      \multicolumn{4}{|c|}{\color{gray} TCP, UDP} \\ \hline
    \end{tabular}
  \end{center}

  \begin{itemize}
    \item Handshake et ChangeCipherSpec (CCS) pour établir la connection
    \item Alert rarement : problèmes, fin de connection
    \item Record pour tout : encapsule, chiffre et authentifie les messages
  \end{itemize}
\end{frame}

\begin{frame}{Structure globale}
  \begin{description}
    \item[Type] un octet : CCS = 20, Alert = 21, HS = 22, App = 23
    \item[Version] deux octets : 0x03 0x03 pour TLS 1.2
    \item[Length] deux octets : longueur du reste en gros-boutiste (limitée à
      $2^{14}$ octets, soit 16 Ko).
  \end{description}

  \begin{center}
    \renewcommand\arraystretch{1.2}
    \begin{tabular}{|c|c|c|c|c|p{5em}|}
      \hline
      T & VM & Vm & L1 & L2 & \centering ... \cr
      \hline
    \end{tabular}
  \end{center}

  Trois types de chiffrement-authentification :
  \begin{enumerate}
    \item Par flot (y compris NULL)
    \item Par bloc avec CBC
    \item AEAD = authenticated encryption with additional data (seulement TLS
      1.2)
  \end{enumerate}
\end{frame}

\begin{frame}{Compression (1)}
  La couche \emph{record} peut compresser avant de chiffrer. Ceci n'est
  \emph{plus} recommandé.

  \begin{block}{L'attaque CRIME}
    \begin{itemize}
      \item Compression Ratio Info-leak Made Easy
      \item Idée de 2002, exploit (public) en 2012
      \item Application possible : fuite de cookies sécurisé
        \begin{enumerate}
          \item On connait \texttt{Cookie secret=}
          \item Pour chaque x, on ajoute \texttt{Cookie secret=x} et on mesure
            la longueur du chiffré
          \item La valeur qui donne un résultat plus court est la bonne
          \item On itère pour les caractères suivants
        \end{enumerate}
      \item Vole un cookie en 30 secondes
    \end{itemize}
  \end{block}

  Dans TLS 1.3, la compression ne sera plus disponible.
\end{frame}

\begin{frame}{Compression (2)}
  \begin{block}{Conditions pour CRIME}
    \begin{enumerate}
      \item L'attaquant peut injecter du texte clair (JS, actif)
      \item L'attaquant peut observer le chiffré (réseau, passif)
      \item La compresssion TLS doit être activée
    \end{enumerate}
  \end{block}

  \begin{block}{Variantes}
    \begin{description}
      \item[TIME] Timing Info-leak Make Easy : supprime la condition 2, mesure
        le temps à la place
      \item[BREACH] Browser Reconnaissance \& Exfiltration via Adaptive
        Compression of Hypertext : exploite la compression HTTP, plus
        répandue.
    \end{description}
  \end{block}
\end{frame}


\section[Flot]{Chiffrement par flot avec RC4}
\tocsect

\begin{frame}{Chiffrement par flot (RC4)}
  \begin{block}{Chiffrement authentifié}
    \begin{itemize}
      \item MAC = HMAC(\no séquence, type, version, longueur, message)
      \item Chiffré = RC4(message, MAC)
      \item Envoyé = Chiffré
      \item L'état RC4 est conservé entre les messages (problème pour DTLS)
    \end{itemize}
  \end{block}

  \begin{block}{Problème : RC4 n'est plus sûr}
    \begin{itemize}
      \item 2013 : RC4 utilisé pour plus de 60~\% des connections HTTPS
      \item 2013 : Royal Holloway, presque pratique ($2^{24}$ chiffrés)
      \item début 2015 : RFC 7465 MUST NOT RC4
      \item mi 2015 : Bar-mitzvah, NOMORE, utilisable en pratique
      \item courant 2015 : RC4 retiré des navigateurs courants
    \end{itemize}
  \end{block}
\end{frame}

\begin{frame}[containsverbatim]{Aparté : comment vérifier un MAC}
  \begin{block}{Méthode naturelle}
    \begin{enumerate}
      \item Calculer la bonne valeur
      \item Comparer avec la valeur reçue avec \texttt{memcmp()}
    \end{enumerate}
  \end{block}

  \begin{block}{Attaque par timing}
    \begin{itemize}
      \item La durée d'exécution de \texttt{memcmp()} est proportionnelle à la
        longueur du préfixe correct
      \item On brute-force octet par octet
    \end{itemize}
  \end{block}

  \begin{block}{Une solution}
    \begin{Verbatim}[gobble=4]
      unsigned char diff = 0;
      for (size_t i = 0; i < len; i++)
          diff |= a[i] ^ b[i];
    \end{Verbatim}
  \end{block}
\end{frame}


\section[CBC]{Chiffrement par bloc avec CBC}
\tocsect

\begin{frame}{Rappel : CBC}
  % wget https://upload.wikimedia.org/wikipedia/commons/8/80/CBC_encryption.svg
  % inkscape -D -z --file=CBC_encryption.svg --export-pdf=cbc-enc.pdf
  \includegraphics<1>[width=\textwidth]{cbc-enc}

  % wget https://upload.wikimedia.org/wikipedia/commons/2/2a/CBC_decryption.svg
  % inkscape -D -z --file=CBC_decryption.svg --export-pdf=cbc-dec.pdf
  \includegraphics<2>[width=\textwidth]{cbc-dec}

  \medskip

  (Crédit image : Wikipédia.)
\end{frame}

\begin{frame}{Chiffrement authentifié avec CBC}
  \begin{block}{TLS : MAC then Encrypt (MtE)}
    \begin{itemize}
      \item MAC = HMAC(métadonnées, message)
      \item Sortie = AES-CBC(message, MAC, padding)
    \end{itemize}
  \end{block}

  \begin{block}{Encrypt and MAC}
    \begin{itemize}
      \item MAC = HMAC(métadonnées, message)
      \item Sortie = AES-CBC(message, padding), MAC
    \end{itemize}
  \end{block}

  \begin{block}{Encrypt then MAC (EtM)}
    \begin{itemize}
      \item Chiffré = AES-CBC(message, padding)
      \item MAC = HMAC(métadonnées, chiffré)
      \item Sortie = Chiffré, MAC
    \end{itemize}
  \end{block}
\end{frame}

\begin{frame}{Détails sur le padding}
  \begin{block}{Principe}
    \begin{itemize}
      \item $L \in \{ 0, \dots, 255 \}$ tel que \( L + 1 + l(m) = 0 \mod b \).
      \item En pratique, \( L \in \{ 0, \dots, b - 1 \} \) et \( b = 16 \)
        pour AES.
      \item Valeur avec SSL 3.0 : $L$ octets quelconques suivis d'un octet de
        valeur $L$.
      \item Valeur avec TLS 1.x : $L + 1$ octets de valeur $L$.
    \end{itemize}
  \end{block}

  \begin{block}{Exemple}
    \begin{itemize}
      \item Message de longueur 4.
      \item Algos utilisés : AES et HMAC-SHA1.
      \item Longueur totale avant padding : 24 octets.
      \item Longueur typique du padding : 8 octets.
      \item Valeur du padding TLS : \texttt{07 07 07 07 07 07 07 07}.
    \end{itemize}
  \end{block}
\end{frame}

\begin{frame}{Un oracle de padding : POODLE (1)}
  \begin{block}{Observation clé}
    \begin{itemize}
      \item Supposons qu'il y a un bloc complet de padding, chiffré $C_n$.
      \item On remplace $C_n$ par un $C'$ quelconque.
      \item Alors le message est accepté ssi $\text{AES}^{-1}(C') \oplus
        C_{n-1}$ se termine par un octet de valeur $15$. (Exercice.)
      \item Si le message est accepté (une fois sur 256) on connait le dernier
        octet de $\text{AES}^{-1}(C')$.
      \item La victime innocente s'est comportée comme un \emph{oracle}.
    \end{itemize}
  \end{block}

  \begin{block}{Contexte}
    \begin{itemize}
      \item On cherche à voler un cookie
      \item On peut générer des requêtes contenant le cookie
      \item On peut modifier la requête chiffrée sur le chemin
    \end{itemize}
  \end{block}
\end{frame}

\begin{frame}[containsverbatim]{Un oracle de padding : POODLE (2)}
  \begin{Verbatim}[gobble=4]
    POST /path Cookie: name=?????\r\n\r\n body MAC padding
  \end{Verbatim}
  \begin{enumerate}
    \item On ajuste \texttt{path} et \texttt{body} pour que:
      \begin{itemize}
        \item L'octet visé soit le dernier du bloc $i$.
        \item Le padding soit de longueur un bloc exactement.
      \end{itemize}
    \item Le client innocent chiffre en $C_1, \dots, C_i, \dots, C_n$.
    \item On envoie au serveur $C_1, \dots, C_i, \dots, C_{n-1}, C_i$.
    \item Le serveur innocent nous donne une erreur, ou une fois sur 256, la
      valeur du dernier octet de $\text{AES}^{-1}(C_i)$.
    \item On XORe avec le dernier octet de $C_{i-1}$ pour trouver la valeur de
      l'octet visé.
    \item On passe à un autre octet jusqu'à avoir tout le cookie.
  \end{enumerate}
\end{frame}

\begin{frame}{Un oracle de padding : POODLE (3)}
  \begin{block}{Analyse}
    \begin{itemize}
      \item Le client innocent connait le cookie secret et la clé de
        chiffrement.
      \item Le serveur innocent connaît la clé de chiffrement.
      \item Le serveur nous donne 8 bits du secret avec probabilité $2^{-8}$.
    \end{itemize}
  \end{block}

  \begin{block}{Historique}
    \begin{itemize}
      \item 1999 : problème résolu dans TLS 1.0
      \item 2003 : principe de l'oracle connu
      \item 2015 : exploit pratique publié, accélère la mort de SSL 3.0
        (Padding Oracle On Downgraded Legacy Encryption)
    \end{itemize}
  \end{block}
\end{frame}

\begin{frame}{L'oracle de Vaudenay et Lucky 13}
  \begin{block}{Vaudenay}
    \begin{itemize}
      \item Premier oracle de padding contre CBC publié (2002)
      \item POODLE en est une variante plus simple
      \item Fonctionne \emph{presque} contre TLS 1.0+ (alertes chiffrées)
      \item Première variante par timing : 2003
      \item Contre-mesure 1 : toujours vérifier le MAC
    \end{itemize}
  \end{block}

  \begin{block}{Lucky 13}
    \begin{itemize}
      \item Autre variante utilisant le timing et un effet de seuil
      \item Contre-mesure 2 : toujours MACer la même longueur
      \item Contre-mesure délicate à implémenter
        (cf \emph{Lucky 13 strikes back} : cache cross-VM)
    \end{itemize}
  \end{block}
\end{frame}

\begin{frame}{Encrypt-then-MAC}
  \begin{block}{Réparer CBC}
    \begin{itemize}
      \item Cause commune à l'attaque de Vaudenay, POODLE et Lucky 13 : le
        padding n'est pas authentifié.
      \item Solution : utiliser Encrypt-then-MAC (RFC 7366)
    \end{itemize}
  \end{block}

  \begin{block}{Historique}
    \begin{itemize}
      \item TLS 1.0 arrête POODLE
      \item Le RFC 5246 recommande la contre-mesure 1, et envisage la 2 mais
        pense qu'elle n'est pas nécessaire
      \item CBC avec EtM comme dans TLS 1.0 a une preuve de sécurité, qui ne
        prend pas en compte les canaux auxiliaires
    \end{itemize}
  \end{block}
\end{frame}

\begin{frame}{Vecteur d'initialisation}
  \includegraphics<1>[width=\textwidth]{cbc-enc}
  \begin{description}
    \item[TLS 1.0-] Le tout premier est calculé au même moment que la clé,
      puis on réutilise le dernier bloc chiffré
    \item[TLS 1.1+] Chaque IV est aléatoire et transmis explicitement
  \end{description}
\end{frame}

\begin{frame}{Blockwise attack}
  \begin{block}{Principe}
    \begin{enumerate}
      \item Rappel/notations $C_i = E( C_{i-1} \oplus P_i )$ ; on suppose $i <
        j$.
      \item En conséquence $C_i = C_j \Longleftrightarrow C_{i-1} \oplus P_i =
        C_{j-1} \oplus P_j$.
      \item Supposons que pour certains $j$, l'attaquant peut choisir $P_j$
        à un moment où $C_k$ est connu pour tout $k < j$.
      \item L'attaquant veut tester un candidat $P'_i$ pour la valeur de $P_i$.
      \item On choisit $P_j = P'_i \oplus C_{i-1} \oplus C_{j-i}$ et
        on regarde si $C_i = C_j$.
    \end{enumerate}
  \end{block}

  \begin{block}{Analyse}
    \begin{itemize}
      \item Publiée en 2004, corrigée en 2006 dans TLS 1.1 (point 3).
      \item Il faut deviner un bloc entier : la force brute ne marche pas
      \item Contourne la preuve de sécurité de CBC (CPA2 \textrightarrow{}
        BCPA2)
    \end{itemize}
  \end{block}
\end{frame}

\begin{frame}{BEAST}
  \begin{itemize}
    \item 2011 : Browser Exploit Against SSL/TLS
    \item Principe : aligner les données pour n'avoir qu'un octet inconnu dans
      le bloc cible, puis le brute-forcer et itérer.
    \item Résolution correcte : passer à TLS 1.1 (70~\% 4 ans après)
    \item Conseil à l'époque : utiliser RC4 à la place\dots oups !
    \item Contre-mesure pratique : $1/n-1$ record splitting
    \item Premier exploit médiatisé utilisant un MitB, a ouvert la voie
  \end{itemize}
\end{frame}


\section[AEAD]{Chiffrement authentifié}
\tocsect

\begin{frame}{Authenticated Encryption with Additional Data}
  \begin{block}{Motivation}
    \begin{itemize}
      \item Le choix de MtE au lieu de EtM est un problème de
        \emph{composition} de primitive cryptographiques (chiffre et MAC)
      \item Problème de crypto, pas de chaque protocole
      \item Une API unifiée permet ceci et offre plus de liberté
    \end{itemize}
  \end{block}

  \begin{description}
    \item[Avant] Plusieurs étapes définies par TLS :\\
      MAC = HMAC(clé HMAC, métadonnées, message)\\
      padding = cf plus haut\\
      Sortie = AES-CBC(clé AES, IV, message, MAC, padding)\\
    \item[Après] Sortie = AEAD(clé, nonce, message, métadonnés)
  \end{description}

  Disponible dans TLS depuis TLS 1.2.
\end{frame}

\begin{frame}{L'API AEAD (RFC 5116)}
  \begin{block}{Entrées}
    \begin{itemize}
      \item Une clé secret unique (générée uniformément au hasard)
      \item Un nonce : ne doit \emph{jamais} être réutilisé avec la même clé
      \item Le message à (dé)chiffrer
      \item Des données supplémentaires authentifiées mais pas chiffrées
    \end{itemize}
  \end{block}
  \begin{block}{Sortie}
    \begin{itemize}
      \item Soit une erreur (le chiffré n'est pas authentique)
      \item Soit le message (dé)chiffré
    \end{itemize}
  \end{block}
\end{frame}

\begin{frame}{En pratique}
  \begin{block}{AES-GCM}
    \begin{itemize}
      \item Galois/Counter Mode (2005)
      \item Chiffrement avec AES en mode compteur
      \item Authentification utilisant la multiplication dans un corps fini
      \item Standard, preuve de sécurité, recommandé (Suite B, NIST, IETF).
      \item Potentiellement très rapide, encore plus avec AES-NI
    \end{itemize}
  \end{block}

  \begin{block}{AES-CCM}
    Comparable, moins rapide mais implémentation plus compacte.
  \end{block}

  \begin{block}{Nonce}
    En pratique, utiliser le compteur de \emph{records} TLS.
  \end{block}
\end{frame}


\section[Point]{Faisons le point}
\tocsect

\begin{frame}{Historique}
  \begin{tabular}{rlll}
    \toprule
    Année & Nom & Conditions & exploit \\
    \midrule
    2002 & Vaudenay   & TLS 1.0- + accès aux alertes    & non \\
    2002 & CBCATT     & TLS 1.0- + CPA + chance         & non \\
    2002 & Compression & ?                              & non \\
    2003 & CBCTIME    & CPA                             & non \\
    2011 & BEAST      & MitB + MitM                     & oui \\
    2012 & CRIME      & compr TLS + MitB + écoute       & oui \\
    2013 & TIME       & compr TLS + MitB + temps        & oui \\
    2013 & BREACH     & compr HTTP + MitB + écoute      & oui \\
    2013 & RC4 biases & session multiples               & partiel \\
    2015 & POODLE     & SSL 3.0, MitB + MitM            & partiel \\
    2015 & RC4        & sessions multiples              & oui \\
    \bottomrule
  \end{tabular}
\end{frame}

\begin{frame}{Références}
  \begin{block}{Méta-références}
    \begin{itemize}
      \item \url{https://tools.ietf.org/html/rfc7457}
      \item \url{https://www.ietf.org/proceedings/89/slides/slides-89-irtfopen-1.pdf}
      \item \url{https://eprint.iacr.org/2013/049.pdf} section III
    \end{itemize}
  \end{block}

  \begin{block}{Références spécifiques complémentaires}
    \begin{itemize}
      \item \url{https://www.openssl.org/~bodo/ssl-poodle.pdf}
      \item \url{https://www.rc4nomore.com/}
      \item \url{http://www.imperva.com/docs/HII_Attacking_SSL_when_using_RC4.pdf}
    \end{itemize}
  \end{block}
\end{frame}

\end{document}
